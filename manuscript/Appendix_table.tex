% Options for packages loaded elsewhere
\PassOptionsToPackage{unicode}{hyperref}
\PassOptionsToPackage{hyphens}{url}
%
\documentclass[
  landscape]{article}
\usepackage{amsmath,amssymb}
\usepackage{lmodern}
\usepackage{ifxetex,ifluatex}
\ifnum 0\ifxetex 1\fi\ifluatex 1\fi=0 % if pdftex
  \usepackage[T1]{fontenc}
  \usepackage[utf8]{inputenc}
  \usepackage{textcomp} % provide euro and other symbols
\else % if luatex or xetex
  \usepackage{unicode-math}
  \defaultfontfeatures{Scale=MatchLowercase}
  \defaultfontfeatures[\rmfamily]{Ligatures=TeX,Scale=1}
\fi
% Use upquote if available, for straight quotes in verbatim environments
\IfFileExists{upquote.sty}{\usepackage{upquote}}{}
\IfFileExists{microtype.sty}{% use microtype if available
  \usepackage[]{microtype}
  \UseMicrotypeSet[protrusion]{basicmath} % disable protrusion for tt fonts
}{}
\makeatletter
\@ifundefined{KOMAClassName}{% if non-KOMA class
  \IfFileExists{parskip.sty}{%
    \usepackage{parskip}
  }{% else
    \setlength{\parindent}{0pt}
    \setlength{\parskip}{6pt plus 2pt minus 1pt}}
}{% if KOMA class
  \KOMAoptions{parskip=half}}
\makeatother
\usepackage{xcolor}
\IfFileExists{xurl.sty}{\usepackage{xurl}}{} % add URL line breaks if available
\IfFileExists{bookmark.sty}{\usepackage{bookmark}}{\usepackage{hyperref}}
\hypersetup{
  pdftitle={An Introductory Tutorial on Cohort State-Transition Models in R Using a Cost-Effectiveness Analysis Example},
  pdfauthor={Fernando Alarid-Escudero, PhD; Eline Krijkamp, MSc; Eva A. Enns, PhD; Alan Yang, MSc; Myriam G.M. Hunink, PhD\^{}\textbackslash dagger; Petros Pechlivanoglou, PhD; Hawre Jalal, MD, PhD},
  hidelinks,
  pdfcreator={LaTeX via pandoc}}
\urlstyle{same} % disable monospaced font for URLs
\usepackage[margin=1in]{geometry}
\usepackage{longtable,booktabs,array}
\usepackage{calc} % for calculating minipage widths
% Correct order of tables after \paragraph or \subparagraph
\usepackage{etoolbox}
\makeatletter
\patchcmd\longtable{\par}{\if@noskipsec\mbox{}\fi\par}{}{}
\makeatother
% Allow footnotes in longtable head/foot
\IfFileExists{footnotehyper.sty}{\usepackage{footnotehyper}}{\usepackage{footnote}}
\makesavenoteenv{longtable}
\usepackage{graphicx}
\makeatletter
\def\maxwidth{\ifdim\Gin@nat@width>\linewidth\linewidth\else\Gin@nat@width\fi}
\def\maxheight{\ifdim\Gin@nat@height>\textheight\textheight\else\Gin@nat@height\fi}
\makeatother
% Scale images if necessary, so that they will not overflow the page
% margins by default, and it is still possible to overwrite the defaults
% using explicit options in \includegraphics[width, height, ...]{}
\setkeys{Gin}{width=\maxwidth,height=\maxheight,keepaspectratio}
% Set default figure placement to htbp
\makeatletter
\def\fps@figure{htbp}
\makeatother
\setlength{\emergencystretch}{3em} % prevent overfull lines
\providecommand{\tightlist}{%
  \setlength{\itemsep}{0pt}\setlength{\parskip}{0pt}}
\setcounter{secnumdepth}{-\maxdimen} % remove section numbering
\ifluatex
  \usepackage{selnolig}  % disable illegal ligatures
\fi

\title{An Introductory Tutorial on Cohort State-Transition Models in R
Using a Cost-Effectiveness Analysis Example}
\usepackage{etoolbox}
\makeatletter
\providecommand{\subtitle}[1]{% add subtitle to \maketitle
  \apptocmd{\@title}{\par {\large #1 \par}}{}{}
}
\makeatother
\subtitle{Appendix}
\author{Fernando Alarid-Escudero, PhD\footnote{Division of Public
  Administration, Center for Research and Teaching in Economics (CIDE),
  Aguascalientes, AGS, Mexico} \and Eline Krijkamp,
MSc\footnote{Department of Epidemiology and Department of Radiology,
  Erasmus University Medical Center, Rotterdam, The Netherlands} \and Eva
A. Enns, PhD\footnote{Division of Health Policy and Management,
  University of Minnesota School of Public Health, Minneapolis, MN, USA} \and Alan
Yang, MSc\footnote{The Hospital for Sick Children, Toronto} \and Myriam
G.M. Hunink, PhD\(^\dagger\)\footnote{Center for Health Decision
  Sciences, Harvard T.H. Chan School of Public Health, Boston, USA} \and Petros
Pechlivanoglou, PhD\footnote{The Hospital for Sick Children, Toronto and
  University of Toronto, Toronto, Ontario, Canada} \and Hawre Jalal, MD,
PhD\footnote{University of Pittsburgh, Pittsburgh, PA, USA}}
\date{2021-09-01}

\begin{document}
\maketitle

\hypertarget{cohort-tutorial-model-components}{%
\subsection{Cohort tutorial model
components}\label{cohort-tutorial-model-components}}

\hypertarget{table-i}{%
\subsubsection{Table I}\label{table-i}}

This table contains an overview of the key model components used in the
code for the Sick-Sicker example from the
\href{http://darthworkgroup.com/}{DARTH} manuscript:
\href{https://arxiv.org/abs/2001.07824}{``An Introductory Tutorial to
Cohort State-Transition Models in R''}. The first column gives the
mathematical notation for some of the model components that are used in
the equations in the manuscript. The second column gives a description
of the model component with the R name in the third column. The forth
gives the data structure, e.g.~scalar, list, vector, matrix etc, with
the according dimensions of this data structure in the fifth column. The
final column indicated the type of data that is stored in the data
structure, e.g.~numeric (5.2,6.3,7.4), category (A,B,C), integer
(5,6,7), logical (TRUE, FALSE).

\begin{longtable}[]{@{}
  >{\raggedright\arraybackslash}p{(\columnwidth - 10\tabcolsep) * \real{0.10}}
  >{\raggedright\arraybackslash}p{(\columnwidth - 10\tabcolsep) * \real{0.33}}
  >{\raggedright\arraybackslash}p{(\columnwidth - 10\tabcolsep) * \real{0.14}}
  >{\raggedright\arraybackslash}p{(\columnwidth - 10\tabcolsep) * \real{0.14}}
  >{\raggedright\arraybackslash}p{(\columnwidth - 10\tabcolsep) * \real{0.17}}
  >{\raggedright\arraybackslash}p{(\columnwidth - 10\tabcolsep) * \real{0.12}}@{}}
\toprule
Parameter & Description & R name & Data structure & Dimensions & Data
type \\
\midrule
\endhead
\(n_t\) & Time horizon & \texttt{n\_cycles} & scalar & & numeric \\
& Cycle length & \texttt{cycle\_length} & scalar & & numeric \\
\(v_s\) & Names of the health states & \texttt{v\_names\_states} &
vector & \texttt{n\_states} x 1 & character \\
\(n_s\) & Number of health states & \texttt{n\_states} & scalar & &
numeric \\
\(v_{str}\) & Names of the strategies & \texttt{v\_names\_str} & scalar
& & character \\
\(n_{str}\) & Number of strategies & \texttt{n\_str} & scalar & &
character \\
\(d_c\) & Discount rate for costs & \texttt{d\_c} & scalar & &
numeric \\
\(d_e\) & Discount rate for effects & \texttt{d\_e} & scalar & &
numeric \\
\(\mathbf{d_c}\) & Discount weights vector for costs & \texttt{v\_dwc} &
vector & (\texttt{n\_t} x 1 ) + 1 & numeric \\
\(\mathbf{d_e}\) & Discount weights vector for effects & \texttt{v\_dwe}
& vector & (\texttt{n\_t} x 1 ) + 1 & numeric \\
& Sequence of cycle numbers & \texttt{v\_cycles} & vector &
(\texttt{n\_t} x 1 ) + 1 & numeric \\
\(\mathbf{wcc}\) & Within-cycle correction weights & \texttt{v\_wcc} &
vector & (\texttt{n\_t} x 1 ) + 1 & numeric \\
\(age_{_0}\) & Age at baseline & \texttt{n\_age\_init} & scalar & &
numeric \\
\(age\) & Maximum age of follow up & \texttt{n\_age\_max} & scalar & &
numeric \\
\(M\) & Cohort trace & \texttt{m\_M} & matrix & (\texttt{n\_t} + 1) x
\texttt{n\_states} & numeric \\
\(m_0\) & Initial state vector & \texttt{v\_m\_init} & vector & 1 x
\texttt{n\_states} & numeric \\
\(m_t\) & State vector in cycle \(t\) & \texttt{v\_mt} & vector & 1 x
\texttt{n\_states} & numeric \\
& & & & & \\
& \textbf{Transition probabilities and rates} & & & & \\
\(p_{[H,S1]}\) & From Healthy to Sick conditional on surviving &
\texttt{p\_HS1} & scalar & & numeric \\
\(p_{[S1,H]}\) & From Sick to Healthy conditional on surviving &
\texttt{p\_S1H} & scalar & & numeric \\
\(p_{[S1,S2]}\) & From Sick to Sicker conditional on surviving &
\texttt{p\_S1S2} & scalar & & numeric \\
\(p_{[S1,S2]_{trtB}}\) & From Sicker to Sick under treatment B
conditional on surviving & \texttt{p\_S1S2\_trtB} & scalar & &
numeric \\
\(r_{[H,D]}\) & Constant rate of dying when Healthy (all-cause mortality
rate) & \texttt{r\_HD} & scalar & & numeric \\
\(r_{[S1,S2]}\) & Constant rate of becoming Sicker when Sick &
\texttt{r\_S1S2} & scalar & & numeric \\
\(r_{[S1,S2]_{trtB}}\) & Constant rate of becoming Sicker when Sick for
treatment B & \texttt{r\_S1S2\_trtB} & scalar & & numeric \\
\(hr_{[S1,H]}\) & Hazard ratio of death in Sick vs Healthy &
\texttt{hr\_S1} & scalar & & numeric \\
\(hr_{[S2,H]}\) & Hazard ratio of death in Sicker vs Healthy &
\texttt{hr\_S2} & scalar & & numeric \\
\(hr_{[S1,S2]_{trtB}}\) & Hazard ratio of becoming Sicker when Sick
under treatment B & \texttt{hr\_S1S2\_trtB} & scalar & & numeric \\
\(P\) & Time-independent transition probability matrix* & \texttt{m\_P}
& matrix & \texttt{n\_states} x \texttt{n\_states} & numeric \\
& * \texttt{\_trtX} is used to specify for which strategy the transition
probability matrix is & & & & \\
& & & & & \\
& \textbf{Annual costs} & & & & \\
& Healthy individuals & \texttt{c\_H} & scalar & & numeric \\
& Sick individuals in Sick & \texttt{c\_S1} & scalar & & numeric \\
& Sick individuals in Sicker & \texttt{c\_S2} & scalar & & numeric \\
& Dead individuals & \texttt{c\_D} & scalar & & numeric \\
& Additional costs treatment A & \texttt{c\_trtA} & scalar & &
numeric \\
& Additional costs treatment B & \texttt{c\_trtB} & scalar & &
numeric \\
& Vector of state costs for a strategy & \texttt{v\_c\_str} & vector & 1
x \texttt{n\_states} & numeric \\
& list that stores the vectors of state costs for each strategy &
\texttt{l\_c} & list & & numeric \\
& & & & & \\
& \textbf{Utility weights} & & & & \\
& Healthy individuals & \texttt{u\_H} & scalar & & numeric \\
& Sick individuals in Sick & \texttt{u\_S1} & scalar & & numeric \\
& Sick individuals in Sicker & \texttt{u\_S2} & scalar & & numeric \\
& Dead individuals & \texttt{u\_D} & scalar & & numeric \\
& Treated with treatment A & \texttt{u\_trtA} & scalar & & numeric \\
& Vector of state utilities for a strategy & \texttt{v\_u\_str} & vector
& 1 x \texttt{n\_states} & numeric \\
& List that stores the vectors of state utilities for each strategy &
\texttt{l\_u} & list & & numeric \\
& & & & & \\
& \textbf{Outcome structures} & & & & \\
& Expected QALYs per cycle under a strategy & \texttt{v\_qaly\_str} &
vector & 1 x (\texttt{n\_t} + 1) & numeric \\
& Expected costs per cycle under a strategy & \texttt{v\_cost\_str} &
vector & 1 x (\texttt{n\_t} + 1) & numeric \\
& Vector of expected discounted QALYs for each strategy &
\texttt{v\_tot\_qaly} & vector & 1 x \texttt{n\_states} & numeric \\
& Vector of expected discounted costs for each strategy &
\texttt{v\_tot\_cost} & vector & 1 x \texttt{n\_states} & numeric \\
& Summary matrix with costs and QALYS per strategy &
\texttt{m\_outcomes} & table & \texttt{n\_states} x 2 & \\
& Summary of the model outcomes & \texttt{df\_cea} & data frame & & \\
& Summary of the model outcomes & \texttt{table\_cea} & table & & \\
& & & & & \\
& \textbf{Probabilistic analysis structures} & & & & \\
& Number of PSA iterations & \texttt{n\_sim} & scalar & & numeric \\
& List that stores all the values of the input parameters &
\texttt{l\_params\_all} & list & & numeric \\
& Data frame with the parameter values for each PSA iteration &
\texttt{df\_psa\_input} & data frame & & numeric \\
& Vector with the names of all the input parameters &
\texttt{v\_names\_params} & vector & & character \\
& List with the model outcomes of the PSA for all strategies &
\texttt{l\_psa} & list & & numeric \\
& Vector with a sequence of relevant willingness-to-pay values &
\texttt{v\_wtp} & vector & & numeric \\
& Data frame to store expected costs and effects for each strategy from
the PSA & \texttt{df\_out\_ce\_psa} & data frame & & numeric \\
& Data frame to store incremental cost-effectiveness ratios (ICERs) from
the PSA & \texttt{df\_cea\_psa} & data frame & & numeric \\
& For more details about the PSA structures read \texttt{dampack}'s
vignettes & & & & \\
\bottomrule
\end{longtable}

\end{document}
